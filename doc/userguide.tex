\input opmac

\margins/1 a4 (3, 3, 3, 3)cm
\typosize [13/15]
\activettchar|

\def\nb{|netbufs|}

\def\hnote#1#2{
	\medskip
	\hrule
	\medskip
	\par\noindent
	{\bf #1}
	\noindent
	\par\noindent#2
	\medskip
	\hrule
	\medskip
}

\def\img#1#2{
	\bigskip
    \centerline{
		\inspic{#1}
    }
	\caption/f #2
	\bigskip
}

\tit Using Network Buffers

\nonum\notoc\sec Contents
\maketoc

\sec What's this?

\hnote{TL;DR}{%
	\nb\ provide serialization and deserialization features to your application.
	They try to be as flexible as possible without compromising performance too
	much and to be easy to use and reliable.
}

\noindent
Imagine that you have data you need to send over the network or write to a
file\fnote{Or exchange with the rest of the world in any other way.}, and you
need to choose some format for your data, which would ideally be:

\begitems\style -
	* simple to use,
	* fast to read and write,
	* compact, and
	* out of the way.
\enditems

\noindent
You'll likely devise multiple solutions with varying degree of satisfaction:

{\bf Fixed scheme.} The simplest solution would be to serialize all the data
you have in some fixed order without any metadata. Sure enough, this will be
quite compact, encoding and decoding is super fast$\dots$ but terribly fragile.

\img{images/fixed.pdf}{Meteorological data encoded in a fixed binary scheme.}

If you start sending\fnote{`Sending', `writing' and `encoding' are all synonyms
in this guide, just as `receiving', `reading' and `decoding' are.} new data and
don't update the reciever to decode it, it will break. You have to always send
all values (even when they are ``unset'' or ``empty''), otherwise it will break, too.

\noindent
{\bf JSON.} These problems are often relevant, so you may want to
switch to JSON or similar scheme. That certainly fixes the problem, but it also
introduces new ones: the scheme is far from efficient, both in terms of time
and size, mainly because of all the string keys you send.

Also, you may find that the API of most JSON decoders won't really be very
helpful as you try do deserialize your data and you'll either want to write your
own deserialization routines, or at least a lot of boilerplate code.

\bigskip
\begtt
reading: {
    wind: {
        speed: 2,
        gust_speed: 11,
        direction: "NNW",
    }, ...
}
\endtt
\caption/f The same data encoded in JSON.
\bigskip

\noindent
{\bf CBOR.} So it would be better to use a binary scheme after all,
which would be as flexible as JSON, but efficient. Luckily for us, there's
CBOR, which does exactly that.

So you'll find yourself a decent implementation of CBOR, but the API
still doesn't fit your needs! Furthemore, you still have to write all the keys,
and those strings still have a hold on your performance.

\img{images/cbor.pdf}{The same data encoded in CBOR (collapsed strings).}

\noindent
{\bf Network Buffers.} So you'll add smart string de-duplication, a nice API
that fits common usage patterns, you'll make it run fast and write a lot of
tests and documentation. Congratulations, you have just invented \nb!

TODO Add figure (netbuf encoding)

\sec Basic Usage
\sec Examples
\sec API Documentation

\bye
